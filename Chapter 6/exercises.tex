\documentclass[a4paper,12pt]{article}
\usepackage{amsmath}
\usepackage[margin=0.9in]{geometry}
\usepackage{braket}
\usepackage{graphicx}
\usepackage{tikz}
\usepackage{amssymb}
\usepackage{mathtools}
\usetikzlibrary{quantikz}
\usepackage{multicol}
\begin{document}
\subsection*{Exercise 6.1}
The phase shift performs the following,\\
$\ket{0}\rightarrow\ket{0}$\\
$\ket{x}\rightarrow-\ket{x}$ for $x>0$\\
Hence, checking,\\
$(2\ket{0}\bra{0}-I)\ket{0}=2\ket{0}-\ket{0}=\ket{0}$\\
$(2\ket{0}\bra{0}-I)\ket{x}=0-\ket{x}=-\ket{x}$\\
Also, $(2\ket{0}\bra{0}-I)^\dagger(2\ket{0}\bra{0}-I)=I$\\
Therefore, $2\ket{0}\bra{0}-I$ is the phase shift unitary operator.
\subsection*{Exercise 6.2}
$(2\ket{\psi}\bra{\psi}-I)\displaystyle\sum_k\alpha_k\ket{k}=
\displaystyle\sum_k 2\alpha_k\ket{\psi}\braket{\psi|k}-\alpha_k\ket{k}=
\displaystyle\sum_{x_1}\sum_{x_2}\sum_k 2 \frac{\alpha_k}{N}\ket{x_1}\braket{x_2|k}-\alpha_k\ket{k}=
\displaystyle\sum_{x_1}2\braket{\alpha}\ket{x_1}-\sum_k\alpha_k\ket{k}=
\displaystyle 2\braket{\alpha}\sum_{x_1}\ket{x_1}-\sum_k\alpha_k\ket{k}=
\displaystyle 2\braket{\alpha}\sum_{k}\ket{k}-\sum_k\alpha_k\ket{k}=\\
\sum_k[-\alpha_k+2\braket{\alpha}]\ket{k}$\\
Where we have relabeled, $x_1$ to $k$ as both are the basis vectors of the vector space.
\subsection*{Exercise 6.3}
Let $\ket{\psi}=\sqrt{\frac{N-M}{N}}\ket{\alpha}+\sqrt{\frac{M}{N}}\ket{\beta}=
\cos\frac{\theta}{2}\ket{\alpha}+\sin\frac{\theta}{2}\ket{\beta}=\begin{bmatrix}
    \cos\frac{\theta}{2}\\
    \sin\frac{\theta}{2}
\end{bmatrix}$\\
$G\ket{\psi}=\begin{bmatrix}
    \cos\theta&-\sin\theta\\
    \sin\theta&\cos\theta
\end{bmatrix}
\begin{bmatrix}
    \cos\frac{\theta}{2}\\
    \sin\frac{\theta}{2}
\end{bmatrix}=\begin{bmatrix}
    \cos\frac{\theta}{2}\cos\theta-sin\frac{\theta}{2}\sin\theta\\
    \cos\frac{\theta}{2}\sin\theta+sin\frac{\theta}{2}\cos\theta 
\end{bmatrix}=
\begin{bmatrix}
    \cos\frac{3\theta}{2}\\
    \sin\frac{3\theta}{2}
\end{bmatrix}=\\
\cos\frac{3\theta}{2}\ket{\alpha}+\sin\frac{3\theta}{2}\ket{\beta}$\\
Hence, $G$ does indeed implement the Grover iteration.

\subsection*{Exercise 6.4}
\textbf{Input:} (1) a black box oracle $O$ which performs the transformation $O\ket{x}\ket{q}=
\ket{x}\ket{q\oplus f(x)}$, where $f(x)=0$ $\forall 0\leq x<2^n$ except $x_i$ for $1\leq i\leq m$
for which $f(x_i)=1$; (2) $n+1$ qubits in the state $\ket{0}$.\\
\textbf{Output:} One of the $x_i$.\\
\textbf{Runtime:} $O(\sqrt{2^n})$ operations. Succeeds with probability $O(1)$.\\
\textbf{Procedure:}\\
\begin{multicols}{2}
    \begin{description}
        \item[1.] $\ket{0}^{\otimes n}\ket{0}$
        \item[2.] $\rightarrow \displaystyle\frac{1}{\sqrt{2^n}}\sum_{x=0}^{2^n-1}
        \ket{x}\left[\frac{\ket{0}-\ket{1}}{\sqrt{2}}\right]$
        \item[3.] $[2(\ket{\psi}\bra{\psi}-I)O]^R\displaystyle\frac{1}{\sqrt{2^n}}\sum_{x=0}^{2^n-1}
        \ket{x}\left[\frac{\ket{0}-\ket{1}}{\sqrt{2}}\right]\\
        \approx \displaystyle\frac{1}{\sqrt{2^m}}\sum_{i=1}^{m}\ket{x_i}\left[\frac{\ket{0}-\ket{1}}{\sqrt{2}}\right]$
        \item[4.] $\rightarrow x_k$
        \item initial state
        \item apply $H^\otimes n$ to the first $n$ qubits, and $HX$ to the last qubit
        \item apply the Grover iteration $R\approx \lceil\frac{\pi}{4}\sqrt{\frac{2^n}{M}}\rceil$ times
        \item 
        \item measure the first $n$ qubits
    \end{description}
\end{multicols}
\subsection*{Exercise 6.5}
The following circuit can be used,\\
$\begin{quantikz}
   \lstick{$\ket{q}$}&\qw&\ctrl{1}&\qw&\ctrl{1}&\qw\\
   \lstick{$\ket{-}$}&\qw&\gate{Z}&\gate[wires=2]{O}&\gate{Z}&\qw\\
   \lstick{$\ket{x}$}&\qwbundle{n}&\qw& &\qw&\qw
\end{quantikz}$\\
If $\ket{q}$ is set then on the second qubit we have $\ket{-}\rightarrow\ket{+}$, hence
the oracle performs $\ket{x}\ket{+}\rightarrow\ket{x}\ket{+\oplus f(x)}=\ket{x}\ket{+}$,
i.e. no items are marked. If the $\ket{q}$ is not set then the oracle functions as intended, i.e marking
the correct solutions.
\subsection*{Exercise 6.6}
The gates in the dotted box perform,\\
$U=X_1X_2H_2\text{CNOT}_{12}H_2X_1X_2=
\begin{bmatrix}
    0&X\\
    X&0
\end{bmatrix}
\begin{bmatrix}
    H&0\\
    0&H
\end{bmatrix}
\begin{bmatrix}
    I&0\\
    0&X
\end{bmatrix}
\begin{bmatrix}
    H&0\\
    0&H
\end{bmatrix}
\begin{bmatrix}
    0&X\\
    X&0
\end{bmatrix}=\\
\begin{bmatrix}
    0&X\\
    X&0
\end{bmatrix}
\begin{bmatrix}
    H&0\\
    0&H
\end{bmatrix}
\begin{bmatrix}
    0&HX\\
    XHX&0
\end{bmatrix}=
\begin{bmatrix}
    XHXHX&0\\
    0&I
\end{bmatrix}=
\begin{bmatrix}
    -Z&0\\
    0&I
\end{bmatrix}=
\begin{bmatrix}
    -1&0&0&0\\
    0&1&0&0\\
    0&0&1&0\\
    0&0&0&1
\end{bmatrix}=\\
I-2\ket{00}\bra{00}=-(2\ket{00}\bra{00}-I)$
\subsection*{Exercise 6.7}
The oracle is the operator \\
$O=\displaystyle\sum_{y\neq x}\ket{y}\bra{y}I_2+\ket{x}\bra{x}X_2=
(I_1-\ket{x}\bra{x})I_2+\ket{x}\bra{x}X_2=I+\ket{x}\bra{x}(X_2-I_2)$\\
Let $P_2$ denote the phase gate, then the circuit implements,\\
$OPO=(I+\ket{x}\bra{x}(X_2-I_2))P_2(I+\ket{x}\bra{x}(X_2-I_2))=
P+\ket{x}\bra{x}(P_2X_2-P_2+X_2-P_2+P_2X_2-P_2X_2-P_2X_2+P_2+XP_2X)=
P_2+\ket{x}\bra{x}(-P_2+e^{i\Delta t}P_2^\dagger)$\\
Hence, applying the circuit to a state $\ket{y}\ket{0}$, the circuit performs,\\
$e^{i\Delta t}\left(I-\ket{x}\bra{x}+e^{-i\Delta t}\ket{x}\bra{x}\right)\ket{y}\ket{0}=
\left(I-\ket{x}\bra{x}+\displaystyle\sum_{n=0}\frac{(-i\Delta t)^n}{n!}\ket{x}\bra{x}\right)\ket{y}\ket{0}=\\
\left(I+\displaystyle\sum_{n=1}\frac{(-i\Delta t)^n}{n!}\ket{x}\bra{x}\right)\ket{y}\ket{0}=
\left(\displaystyle\sum_{n=0}\frac{(-i\Delta t\ket{x}\bra{x})^n}{n!}\right)\ket{y}\ket{0}=
e^{-i\ket{x}\bra{x}\Delta t}\ket{y}\ket{0}$\\
The first two gates of the second circuit performs the action,\\
$H^{\otimes n}\left(\displaystyle\sum_{y\neq \ket{0}^{\otimes n}}\ket{y}\bra{y}I_2+\ket{0}\bra{0}X_2\right)=
H^{\otimes n}((I_1-\ket{0}\bra{0})I_2+\ket{0}\bra{0}X_2)=H^{\otimes n}(I+\ket{0}\bra{0}(X_2-I_2))$\\
Hence, the total action of the circuit, similar to above is,\\
$H^{\otimes n}(I+\ket{0}\bra{0}(X_2-I_2))P_2(I+\ket{0}\bra{0}(X_2-I_2))H^{\otimes n}=
P_2+H^{\otimes n}\ket{0}\bra{0}H^{\otimes n}(-P_2+e^{i\Delta t}P_2^\dagger)=
P_2+\ket{\psi}\bra{\psi}(-P_2+e^{i\Delta t}P_2^\dagger)$\\
Therefore, as above applying the circuit on a state $\ket{y}\ket{0}$, results in\\
$e^{-i\ket{\psi}\bra{\psi}\Delta t}\ket{y}\ket{0}$
\subsection*{Exercise 6.8}
For an accuracy of $O(\Delta t^r)$ the cumulative error is $O(\Delta t^r \times \sqrt{N}/\Delta t)=
O(\Delta t^{r-1}\sqrt{N})$, hence we choose $\Delta t=\Theta(N^{-1/(2(r-1))})$, therefore the number of steps is
$\frac{t}{\Delta t}=O(\sqrt{N}N^{1/(2(r-1))})=O(N^{r/(2(r-1))})$.
\subsection*{Exercise 6.9}
Using Exercise 4.15 (b),\\
$\displaystyle c_{12}=\cos^2\left(\frac{\Delta}{2}\right)-\sin^2\left(\frac{\Delta}{2}\right)\vec{\psi}.\hat{z}\\
\displaystyle s_{12}\hat{n}_{12}=\sin\left(\frac{\Delta}{2}\right)\cos\left(\frac{\Delta}{2}\right)(\vec{\psi}+\hat{z})+
\sin^2\left(\frac{\Delta}{2}\right)\vec{\psi}\times\hat{z}$\\
Hence,\\
$\displaystyle U(\Delta t)=\left(\cos^2\left(\frac{\Delta}{2}\right)-\sin^2\left(\frac{\Delta}{2}\right)\vec{\psi}.\hat{z}\right)I-
2i\sin\left(\frac{\Delta t}{2}\right)\left(\cos\left(\frac{\Delta}{2}\right)\frac{\vec{\psi}+\hat{z}}{2}+
\sin\left(\frac{\Delta}{2}\right)\frac{\vec{\psi}\times\hat{z}}{2}\right).\hat{\sigma}$
\subsection*{Exercise 6.10}
$\displaystyle\cos\left(\frac{\theta}{2}\right)=1-\frac{2}{N}\sin^2\left(\frac{\Delta t}{2}\right)$\\
$\displaystyle\theta=2\arccos\left(1-\frac{2}{N}\sin^2\left(\frac{\Delta t}{2}\right)\right)$\\
Choose the largest $\Delta t$ such that $\theta=\pi/k$ for $k\in\mathbb{Z}$.
\subsection*{Exercise 6.11}
$H=\displaystyle\sum_{i=1}^m\ket{x_i}\bra{x_i}+\ket{\psi}\bra{\psi}$
\subsection*{Exercise 6.12}
1) Let $\ket{\psi}=\alpha\ket{x}+\beta\ket{y}$ 
$H=\ket{x}\bra{\psi}+\ket{\psi}\bra{x}=\begin{bmatrix}
    2\alpha&\beta\\
    \beta&0
\end{bmatrix}=\alpha I+\alpha Z+\beta X$\\
Hence,\\
$e^{-iHt}\ket{\psi}=e^{-i\alpha t}(\cos{t}\ket{\psi}-i\sin{t}(\alpha Z+\beta X)\ket{\psi})=
e^{-i\alpha t}(\cos{t}\ket{\psi}-i\sin{t}(\alpha^2 \ket{x}-\alpha\beta\ket{y}+\alpha\beta\ket{y}+
\beta^2\ket{x}))=e^{-i\alpha t}(\cos{t}\ket{\psi}-i\sin{t}\ket{x})$\\
Therefore, we can let $t=\pi/2$ to get $\ket{x}$, which is $O(1)$ in time.\\
2) We simulated the Hamiltonian $H$ by alternating application of the Hamiltonians
$H_1=\ket{x}\bra{\psi}$ and $H_2=H_1^\dagger =\ket{\psi}\ket{x}$. Similar to the Hamiltonian simulated
by figures 6.4 and 6.5, each iteration will require $2$ oracle calls, hence we will require
$O(N^{r/(2(r-1))})$ oracle calls for a $\Delta t$ of accuracy $O(\Delta t^r)$.
\subsection*{Exercise 6.13}
$\frac{k}{N}S~B(k,p)$ with $p=\frac{M}{N}$\\
$\text{Var}[\frac{k}{N}S]=\frac{k^2}{N^2}\text{Var}[S]=kp(1-p)=
k\frac{M(N-M)}{N^2}$\\
$\text{Var}[S]=\frac{M(N-M)}{k}$\\
$\Delta S=\sqrt{\frac{M(N-M)}{k}}$\\
From Chebyshev's inequality, $2\Delta S\leq \sqrt{M}$, hence $k\geq 4(N-M)$.
Therefore, $k=\Omega(N)$.
\subsection*{Exercise 6.14}
Assume we sample $A$ values to get an estimate $M+c\sqrt{M}$. The expectation is 
$A\frac{M}{N}$ and the standard deviation $\sqrt{A\frac{M}{N}}$. Therefore, the estimate of
$M$ with 75\% probability is, $M_{est}=\frac{N}{A}(A\frac{M}{N}\pm 2\sqrt{A\frac{M}{N}})=
M+2\sqrt{\frac{N}{A}}\sqrt{M}$\\
Hence,\\
$M+c\sqrt{M}=M_{est}=M+2\sqrt{\frac{N}{A}}\sqrt{M}$\\
$c^2=\frac{4N}{A}$\\
$A=\frac{4N}{c^2}=\Omega(N)$
\subsection*{Exercise 6.15}
From Cauchy-Schwarz,\\
$\displaystyle\left(\sum_x|\braket{x|y}|\times 1\right)^2\leq \sum_x |\braket{x|y}|^2\sum_x 1=N$\\
Hence,\\
$\displaystyle\sum_x||\psi-x||^2=||\psi||^2+||x||^2-2\mathcal{R}(\braket{x|y})=
2N-2\mathcal{R}(\braket{x|y})\geq 2N-2\sqrt{N}$
\subsection*{Exercise 6.16}
Instead of $|\braket{x|\psi_k^x}|^2\geq 1/2$, we have
$\displaystyle\frac{1}{N}\sum_x|\braket{x|\psi_k^x}|^2\geq 1/2$.\\
Hence,\\
$\displaystyle\frac{1}{N}\sum_x||\psi_k^x-x||^2=2-2\frac{1}{N}\sum_x|\braket{x|\psi_k^x}|\leq
2-\sqrt{2}$\\
Therefore,\\
$E_k=\displaystyle\sum_x||\psi_k^x-x||^2\leq (2-\sqrt{2})N$\\
And the rest follows as in the text.
\subsection*{Exercise 6.17}

\subsection*{Exercise 6.18}

\subsection*{Exercise 6.19}
\subsection*{Exercise 6.20}
\end{document}