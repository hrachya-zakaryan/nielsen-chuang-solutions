\documentclass[a4paper,12pt]{article}
\usepackage{amsmath}
\usepackage[margin=0.9in]{geometry}
\usepackage{braket}
\usepackage{graphicx}
\usepackage{quantikz}
\begin{document}
\subsection*{Exercise 10.1}
Let $\ket{\psi}=a\ket{0}+b\ket{1}$ and the initial state be $\ket{\psi_0}=a\ket{000}+b\ket{100}$.\\
Applying a CNOT to the first two qubits we get,\\
$\ket{\psi_1}=a\ket{000}+b\ket{110}$\\
Applying a CNOT to the first and last qubits we get,\\
$\ket{\psi_2}=a\ket{000}+b\ket{111}$
\subsection*{Exercise 10.2}
$P_\pm=\frac{1}{2}(\ket{0}\pm\ket{1})(\bra{0}\pm\bra{1})=
\frac{1}{2}(\ket{0}\bra{0}+\ket{1}\bra{1}\pm\ket{1}\bra{0}\pm\ket{0}\bra{1})
=\frac{1}{2}(I\pm X)$\\
Therefore,\\
$\mathcal{E}(\rho)=(1-2p)\rho+2pP_+\rho P_++2pP_-\rho P_-=
(1-2p)\rho+\frac{1}{2}p(I+X)\rho(I+X)+\frac{1}{2}p(I-X)\rho(I-X)=
(1-2p)\rho+p\rho+pX\rho X=(1-p)\rho+pX\rho X$
\subsection*{Exercise 10.3}
$Z_2Z_3Z_1Z_2=[I\otimes (\ket{00}\bra{00}+\ket{11}\bra{11})-I\otimes (\ket{01}\bra{01}+\ket{10}\bra{10})]
[(\ket{00}\bra{00}+\ket{11}\bra{11})\otimes I-(\ket{01}\bra{01}+\ket{10}\bra{10})\otimes I]=
\underbrace{\ket{000}\bra{000}+\ket{111}\bra{111}}_{P_0}
-(\underbrace{\ket{100}\bra{100}+\ket{011}\bra{011}}_{P_1})\\
+\underbrace{\ket{010}\bra{010}+\ket{101}\bra{101}}_{P_2}
-(\underbrace{\ket{001}\bra{001}+\ket{110}\bra{110}}_{P_3})$
\subsection*{Exercise 10.4}
1)$\ket{000}\bra{000}$, $\ket{111}\bra{111}$: no bit flip\\
$\ket{100}\bra{100}$, $\ket{011}\bra{011}$: first bit flipped\\
$\ket{010}\bra{010}$, $\ket{101}\bra{101}$: second bit flipped\\
$\ket{001}\bra{001}$, $\ket{110}\bra{110}$: third bit flipped\\
2) If our state is $\ket{\psi}=a\ket{000}+b\ket{111}$, then the measurement will
collapse the state into $\ket{000}$ or $\ket{111}$ with probabilities $|a|^2$ or $|b|^2$,
respectively. Hence, only the computational basis states $\ket{000}$ and $\ket{111}$ can 
be corrected.\\
3)Assuming the initial state is $\ket{000}$ the probability that one or fewer bit flips occur is $(1-p)^3+p(1-p)^2$, hence
$F\geq\sqrt{(1-p)^3+p(1-p)^2}$.
\subsection*{Exercise 10.5}
Assuming no more than one error has occurred, $X_1X_2X_3X_4X_5X_6$ will be $1$ if no phase flip occurred
and $-1$ and if one occurred on the first or second block. Identically for $X_4X_5X_6X_7X_8X_9$.
Hence, if both give $-1$ the error is on the second block, otherwise it's on the first block if
$X_1X_2X_3X_4X_5X_6$ gives $-1$ and on the third block if $X_4X_5X_6X_7X_8X_9$ gives $-1$.
If both give $1$ then no error has occurred.
\subsection*{Exercise 10.6}
The eigenvalues of $Z$ are $\pm 1$, hence\\
$Z_1Z_2Z_3(\ket{000}-\ket{111})=\ket{000}-(-1)^3\ket{111}=\ket{000}+\ket{111}$
\subsection*{Exercise 10.7}
Need to prove that $PE_i^\dagger E_jP=\alpha_{ij}P$. $I$ and $X$ are Hermitian, hence suffices
to show for $IX_1$,$II$,$X_1X_1$ and $X_1X_2$.\\
$P\sqrt{(1-p)^3}I\sqrt{p(1-p)^2}X_1P=(1-p)^2\sqrt{p(1-p)}
(\ket{000}\bra{000}+\ket{111}\bra{111})X_1(\ket{000}\bra{000}+\ket{111}\bra{111})=
(1-p)^2\sqrt{p(1-p)}
(\ket{000}\bra{000}+\ket{111}\bra{111})(\ket{100}\bra{000}+\ket{011}\bra{111})=0$\\
$P\sqrt{(1-p)^3}I\sqrt{(1-p)^3}IP=(1-p)^3PP=(1-p)^3P$\\
$P\sqrt{p(1-p)^2}X_1\sqrt{p(1-p)^2}X_1P=p(1-p)^2PIP=p(1-p)^2P$\\
$P\sqrt{p(1-p)^2}X_1\sqrt{p(1-p)^2}X_2=p(1-p)^2
(\ket{000}\bra{000}+\ket{111}\bra{111})(\ket{110}\bra{000}+\ket{001}\bra{111})=0$\\
Hence, the quantum error-correction conditions are satisfied.
\subsection*{Exercise 10.8}
$P=\ket{+++}\bra{+++}+\ket{---}\bra{---}$, hence like in the previous exercise.\\
$PE_i^\dagger E_jP=0$, $i\neq j$\\
$PE_i^\dagger E_jP=P$, $i=j$\\
Hence, the quantum error-correction conditions are satisfied.
\subsection*{Exercise 10.9}
$PIIP=P$\\
$PIP_1P=(\ket{+++}\bra{+++}+\ket{---}\bra{---})(\ket{0}\bra{0}\otimes I\otimes I)
(\ket{+++}\bra{+++}+\ket{---}\bra{---})=
(\ket{+++}\bra{+++}+\ket{---}\bra{---})\frac{1}{\sqrt{2}}(\ket{0++}\bra{+++}+\ket{0--}\bra{---})=
\frac{1}{2}(\ket{+++}\bra{+++}+\ket{---}\bra{---})=\frac{1}{2}P$\\
Identically,\\
$PIQ_1P=\frac{1}{2}P$\\
$PP_1Q_1=0$\\
$PP_1P_1P=PP_1P=\frac{1}{2}P$\\
$PQ_1Q_1P=PQ_1P=\frac{1}{2}P$\\
$PP_1P_2P=(\ket{+++}\bra{+++}+\ket{---}\bra{---})(\ket{0}\bra{0}\otimes \ket{0}\bra{0}\otimes I)
(\ket{+++}\bra{+++}+\ket{---}\bra{---})=
(\ket{+++}\bra{+++}+\ket{---}\bra{---})\frac{1}{2}(\ket{00+}\bra{+++}+\ket{00-}\bra{---})=
\frac{1}{4}(\ket{+++}\bra{+++}+\ket{---}\bra{---})=\frac{1}{4}P$\\
$PP_1Q_2P=(\ket{+++}\bra{+++}+\ket{---}\bra{---})(\ket{0}\bra{0}\otimes \ket{1}\bra{1}\otimes I)
(\ket{+++}\bra{+++}+\ket{---}\bra{---})=
(\ket{+++}\bra{+++}+\ket{---}\bra{---})\frac{1}{2}(\ket{01+}\bra{+++}-\ket{01-}\bra{---})
=\frac{1}{4}(\ket{+++}\bra{+++}+\ket{---}\bra{---})=\frac{1}{4}P$\\
Hence, the quantum error-correction conditions are satisfied.
\subsection*{Exercise 10.10}
$P=\ket{0_L}\bra{0_L}+\ket{1_L}\bra{1_L}$\\
Due to phase and bit flips,\\
$PIX_iP=PIY_iP=PIZ_iP=0$\\
$PIIP=PX_iX_iP=PY_iY_iP=PZ_iZ_iP=P$\\
The $X_i$ and $Y_i$ change the individual qubits, hence if $i\neq j$ $PX_iY_jP=0$, e.g.
for $PX_1Y_2P$ looking at the first triplet, we have\\
$(\bra{000}+\bra{111})i(\ket{110}-\ket{001})=0$\\
$X_iY_i=iZ_i$, hence $PX_iY_iP=0$\\
For $Z_iZ_j$ if $i$ and $j$ belong to different triplets then we have a phase flip on $2$
separate triplets, hence $PZ_iZ_jP=0$. \\
However, if $i$ and $j$ are in the same triplet, then
we apply $2$ phase shifts to the triplet which is equivalent to no change, hence
$PZ_iZ_jP=P$.\\
For $X_iZ_j$ and $Y_iZ_j$ we perform a bit and phase flip, hence for all $i$ and $j$
$PX_iZ_jP=PY_iZ_jP=0$.

\subsection*{Exercise 10.11}
$\mathcal{E}(\rho)=\frac{I}{2}$\\
Consider the operation elements found for the general depolarizing channel in Exercise
8.19 $\{\sqrt{\frac{p}{d}}\ket{i}\bra{j}\}$. Taking $p=1$ and $d=2$, we get 
$\{\frac{1}{2}\ket{0}\bra{0},\frac{1}{2}\ket{1}\bra{1},\frac{1}{2}\ket{0}\bra{1},\frac{1}{2}\ket{1}\bra{0}\}$.
\subsection*{Exercise 10.12}
$F(\ket{0},\mathcal{E}(\ket{0}\bra{0}))=\sqrt{\bra{0}\mathcal{E}(\ket{0}\bra{0})\ket{0}}\\
=\sqrt{\bra{0}((1-p)\ket{0}\bra{0}+\frac{p}{3}(X\ket{0}\bra{0}X+Y\ket{0}\bra{0}+Z\ket{0}\bra{0}Z))\ket{0}}=
\sqrt{1-p+\frac{p}{3}}=\sqrt{1-\frac{2p}{3}}$
As the depolarizing channel is symmetric, for any pure state $\ket{\psi}$,\\ 
$F(\ket{\psi},\mathcal{E}(\ket{\psi}\bra{\psi}))=\sqrt{1-\frac{2p}{3}}$. \\As fidelity is
jointly concave, for any $\rho$ and some $\ket{\psi}$ we have,\\
$F(\rho, \mathcal{E}(\rho))\geq F(\ket{\psi},\mathcal{E}(\ket{\psi}\bra{\psi}))=\sqrt{1-\frac{2p}{3}}$
\subsection*{Exercise 10.13}
Let $\ket{\psi}=a\ket{0}+b\ket{1}$\\
$F(\ket{\psi},\mathcal{E}(\ket{\psi}\bra{\psi}))=\sqrt{\bra{\psi}\mathcal{E}(\ket{\psi}\bra{\psi})\ket{\psi}}
\\
\sqrt{|\bra{\psi}E_0\ket{\psi}|^2+|\bra{\psi}E_1\ket{\psi}|^2}=
\sqrt{||a|^2+|b|^2\sqrt{1-\gamma}|^2+|a|b|^2\sqrt{\gamma}|^2}$\\
Minimum will occur when $a=0$ and $b=1$, hence\\
$F_{min}(\ket{\psi},\mathcal{E}(\ket{\psi}\bra{\psi}))=F(\ket{1},\mathcal{E}(\ket{1}\bra{1}))=\sqrt{1-\gamma}$
\subsection*{Exercise 10.14}
$G$ will be a $rk\times k$ matrix.\\
$G=
rk\left\{
\begin{bmatrix}
    1 & 0 & \ldots & 0\\
    \scriptstyle{r}\vdots & \vdots & \vdots & \vdots\\
    1 & 0 & \ldots & 0\\
    0 & 1 & \ldots & 0\\
    \vdots & \vdots & \vdots & \vdots\\
    0 & 1 & \ldots & 0\\
    \vdots & \vdots & \vdots & \vdots\\
    0 & 0 & \ldots & 1\\
    \vdots & \vdots & \vdots & \vdots\\
    0 & 0 & \ldots & 1
    
\end{bmatrix}\right.$
\newpage
\subsection*{Exercise 10.15}
Let $c_1$ and $c_2$ be columns of $G$. Then\\
$G=[c_1|c_2|G^\prime]\\
G^{\prime\prime}=[c_1|c_1+c_2|G^\prime]$\\
Let $x=(x_1,x_2,\ldots, x_n)$.\\
$Gx=c_1x_1+c_2x_2+\ldots$\\
$G^{\prime\prime}x=c_1x_1+(c_1+c_2)x_2+\ldots$\\
$G^{\prime\prime}x-Gx=c_1x_2\in C$\\
Therefore, as $C$ is linear with $G$ as generator, $G^{\prime\prime}$ is a generator for
$C$ as well, as the difference of the two codes is still in $C$.  
\subsection*{Exercise 10.16}
Let $r_1$ and $r_2$ be rows of $H$. Then\\
$H=\left[\begin{array}{c}
    r_1 \\ \hline
    r_2 \\ \hline
    H^\prime
    \end{array}\right]$\\
$H^{\prime\prime}=\left[\begin{array}{c}
    r_1 \\ \hline
    r_1+r_2 \\ \hline
    H^\prime
    \end{array}\right]$\\
Let $x=(x_1,x_2,\ldots, x_n)$.\\
$Hx=\begin{bmatrix}
    r_1x\\
    r_2x\\
    \vdots
\end{bmatrix}=0$\\
Therefore, $r_1x=r_2x=0$. Hence,\\
$H^{\prime\prime}x=\begin{bmatrix}
    r_1x\\
    r_1x+r_2x\\
    \vdots
\end{bmatrix}=0$\\
Hence, $H^{\prime\prime}$ is a parity check matrix for the same code.
\subsection*{Exercise 10.17}
$y_1=(1,1,1,0,0,0)$, $y_2=(0,0,0,1,1,1)$, hence we can take $y_3$ to $y_6$ as,\\
$y_3=(1,1,0,0,0,0)$\\
$y_4=(1,0,1,0,0,0)$\\
$y_5=(0,0,0,0,1,1)$\\
$y_6=(0,0,0,1,0,1)$\\
Therefore,\\
$H=\begin{bmatrix}
    1&1&0&0&0&0\\
    1&0&1&0&0&0\\
    0&0&0&0&1&1\\
    0&0&0&1&0&1\\
\end{bmatrix}$
\newpage
\subsection*{Exercise 10.18}
Let $x$ be an arbitrary message to be encoded. Then,\\
$y=Gx\in C$\\
Hence,
$HGx=Hy=0$ for $\forall x$\\
Hence,
$HG=0$
\subsection*{Exercise 10.19}
Using that $HG=0$ we have,\\
$HG=\begin{bmatrix}
    a_{11}& a_{12} &\ldots&a_{1k}&1& \ldots &0\\
    \vdots &\vdots &\vdots &\vdots & &\ddots &\\
    a_{(n-k)1}& a_{(n-k)2} &\ldots&a_{(n-k)k}&0 &\ldots &1
\end{bmatrix}
\begin{bmatrix}
    b_{11}&b_{12}&\ldots&b_{1k}\\
    \vdots &\vdots &\vdots &\vdots\\
    b_{n1}&b_{n2}&\ldots&b_{nk}
\end{bmatrix}=0$\\
Hence,\\
$\displaystyle\sum_{i\leq k}a_{1i}b_{i1}+b_{(k+1)1}=0
\ldots
\displaystyle\sum_{i\leq k}a_{(n-k)i}b_{i1}+b_{n1}=0\\
\vdots\\
\displaystyle\sum_{i\leq k}a_{1i}b_{ik}+b_{(k+1)k}=0
\ldots
\displaystyle\sum_{i\leq k}a_{(n-k)i}b_{ik}+b_{nk}=0$\\
We see that for example, taking for $2\leq i \leq k$ $b_{i1}=0$ , $b_{11}=1$ and $b_{(k+1)1}=-a_{11}$
gives a solution.\\
Therefore for $i,j\leq k$ $b_{ij}=\delta_{ij}$ and for $i,j>k$ $b_{ij}=-a_{(i-k)j}$, i.e.\\
$G=\left[\begin{array}{c}
    I_k\\ \hline
    -A
    \end{array}\right]$
\subsection*{Exercise 10.20}
Let $x$ be a codeword such that wt$(x)\leq d-1$. Let $H={c_1|c_2\ldots c_n}$ for code $C$. Consider $Hx$,\\
$Hx=\displaystyle \sum_ic_ix_i$ for $d-1$ columns. Therefore, as any $d-1$ columns are linearly
independent, this sum cannot equal $0$. Hence, $d(C)\geq d$. However, as any $d$ columns
are linearly dependant there exists a codeword $y$ with wt$(y)=d$ such that $Hy=0$. Therefore,
$d(C)=d$.
\subsection*{Exercise 10.21}
The parity check matrix is a $n-k$ by $n$ matrix, hence the maximum number of linearly independent
columns is $n-k$. Therefore, from Exercise 10.20 $n-k\geq d-1$.
\subsection*{Exercise 10.22}
The Hamming parity check matrix is constructed from columns which are all the possible
$n-k$ bit strings, of which there are $2^r-1$ of excluding the $0$ string. Hence, any
two columns will be linearly independent as all are different, however there always will be $3$
linearly dependant columns, e.g. $(1,0,0,\ldots)$, $(0,1,0,\ldots)$ and $(1,1,0,\ldots)$.
Therefore, as per exercise 10.20 the code will have distance $3$. 
\subsection*{Exercise 10.23}

\subsection*{Exercise 10.24}
If $C^\perp\subseteq C$, $\forall x$ ${y=Gx}\in C^\perp$ and $G^T=H^\perp$.
Hence, $\forall x$ $G^TGx=H^\perp y=0$, i.e. $G^TG=0$.\\
If $G^TG=0$, $\forall x$ $G^TGx=H^\perp y=0$, therefore $y\in C^\perp$, hence
$C^\perp\subseteq C$.
\subsection*{Exercise 10.25}
$x=H^Tz_0$\\
If $x\in C^\perp$, \\
$\displaystyle \sum_{y\in C} (-1)^{x.y}=\sum_{z}(-1)^{(H^Tz_0)^TGz}=
\sum_{z}(-1)^{z_0^THGz}=\sum_{z}(-1)^{0}=|C|$\\
If $x\notin C^\perp$,\\
$\displaystyle \sum_{y\in C} (-1)^{x.y}=\sum_{z}(-1)^{x^TGz}$\\
Let, $x^TG=z_1^T$, then\\
$\displaystyle \sum_{y\in C} (-1)^{x.y}=\sum_z(-1)^{z_1.z}$\\
As we're summing over all $z$, $z_1.z=0$ or $1$ both with probability $\frac{1}{2}$.
Hence,\\
$\displaystyle \sum_{y\in C} (-1)^{x.y}=0$
\subsection*{Exercise 10.26}
To perform the transformation $\ket{x}\ket{0}\rightarrow\ket{x}\ket{Hx}$ we perform the
following. Let $\ket{x}=\ket{x_1,x_2,\ldots,x_n}$ and $\ket{0}=\ket{0_1,0_2,\ldots, 0_m}$.
 For each $0_i$,
consider the $i^{th}$ row of $H$ and for each column $j$ which is $1$ apply a
CNOT between $x_j$ and the $0_i$ with $x_j$ the control. After,
applying this for all the qubits of $\ket{0}$ we obtain the desired transformation. As an example
here's the circuit for $H=\begin{bmatrix}
    1 & 1 & 0\\
    0& 1&1
\end{bmatrix}$,\\
\begin{center}
    \includegraphics[scale=0.7]{26.png}
\end{center}
\subsection*{Exercise 10.27}
Consider a bit error $e_1$ and flip error $e_2$. We get,\\
$\displaystyle\frac{1}{\sqrt{|C_2|}}\sum_{y\in C_2}(-1)^{u.y}
(-1)^{(x+y+v).e_2}\ket{x+y+v+e_1}$\\
Applying the parity matrix $H_1$ to $\ket{x+C_2}\ket{0}$ we get,\\
$\displaystyle\frac{1}{\sqrt{|C_2|}}\sum_{y\in C_2}(-1)^{u.y}
(-1)^{(x+y+v).e_2}\ket{x+y+v}\ket{H_1(v+e_1)}$\\
As $v$ is known so is $H_1v$, hence we can calculate the syndrome
$H_1e_1$. Therefore, removing the bit error we get,\\
$\displaystyle\frac{1}{\sqrt{|C_2|}}\sum_{y\in C_2}(-1)^{u.y}
(-1)^{(x+y+v).e_2}\ket{x+y+v}$\\
Applying Hadamard gates to each qubit we get,\\
$\displaystyle\frac{1}{\sqrt{|C_2|2^n}}\sum_z\sum_{y\in C_2}
(-1)^{u.y}(-1)^{(x+y+v).(z+e_2)}\ket{z}=
\frac{1}{\sqrt{|C_2|2^n}}\sum_z\sum_{y\in C_2}
(-1)^{(u+z+e_2).y}(-1)^{(x+v).(z+e_2)}\ket{z}$\\
Let $e_2+z=z^\prime+u$, then we have,\\
$\displaystyle\frac{1}{\sqrt{|C_2|2^n}}\sum_{z^\prime}\sum_{y\in C_2}
(-1)^{z^\prime.y}(-1)^{(x+v).(z^\prime +u)}\ket{z^\prime+e_2+u}$\\
Using Exercise 10.25 we get,\\
$\displaystyle\frac{1}{\sqrt{2^n/|C_2|}}\sum_{z^\prime\in C_2^\perp}
(-1)^{(x+v).(z^\prime +u)}\ket{z^\prime+e_2+u}$\\
Once again by knowing $H_2u$ we calculate the syndrome $H_2e_2$, where
$H_2$ is the parity check matrix for $C_2^\perp$, and hence correct the
error $e_2$ to get,\\
$\displaystyle\frac{1}{\sqrt{2^n/|C_2|}}\sum_{z^\prime\in C_2^\perp}
(-1)^{(x+v).(z^\prime +u)}\ket{z^\prime+u}$\\
Applying the Hadamards again we get,\\
$\displaystyle\frac{1}{\sqrt{|C_2|}}\sum_{y\in C_2}(-1)^{u.y}
\ket{x+y+v}$\\
Hence, this has the same error-correcting properties as the
$CSS(C_1,C_2)$.
\subsection*{Exercise 10.28}
For the $[7,4,3]$ Hamming code we have,\\
$H=
\begin{bmatrix}
    0&0&0&1&1&1&1\\
    0&1&1&0&0&1&1\\
    1&0&1&0&1&0&1
\end{bmatrix}$\\
$HH[C_2]^T=\begin{bmatrix}
    0&0&0&1&1&1&1\\
    0&1&1&0&0&1&1\\
    1&0&1&0&1&0&1
\end{bmatrix}
\begin{bmatrix}
    1&0&0&0\\
    0&1&0&0\\
    0&0&1&0\\
    0&0&0&1\\
    0&1&1&1\\
    1&0&1&1\\
    1&1&0&1\\
\end{bmatrix}=
\begin{bmatrix}
    0&0&0&0\\
    0&0&0&0\\
    0&0&0&0
\end{bmatrix}$\\
Hence, $H[C_2]^T=G[C_1]$.
\subsection*{Exercise 10.29}
Let $\ket{x},\ket{y}\in V_S$, i.e. $\forall g\in S$ 
$g\ket{x}=\ket{x}$ and $g\ket{y}=\ket{y}$. Consider $a\ket{x}+b\ket{y}$
for some $a$ and $b$. As $g$ are linear operators we have,\\
$g(a\ket{x}+b\ket{y})=ag\ket{x}+bg\ket{y}=a\ket{x}+b\ket{y}$\\
Hence, $a\ket{x}+b\ket{y}\in V_S$.\\
Let $\ket{x}\in V_S\implies \forall g\in S$ $g\ket{x}=\ket{x}\implies \forall
g\in S$ $\ket{x}\in V_g\implies \ket{x}\in \displaystyle \bigcap_{g\in S}V_G$ 
\subsection*{Exercise 10.30}
Let $\pm iI\in S$ then as $S$ is a group $(\pm iI)(\pm iI)\in S$, hence $-I\in S$, which
is a contradiction therefore $\pm iI\notin S$.
\subsection*{Exercise 10.31}
If $g_i$ and $g_j$ commute then all the elements of $S$ commute, as
$S$ is generated by the $g_i$'s. If all the elements of $S$ commute
then necessarily $g_i$ and $g_j$ also commute as they're elements
of $S$.
\subsection*{Exercise 10.32}
$g_1\ket{0_L}=\frac{1}{\sqrt{8}}(\ket{0001111}+\ket{1011010}+
\ket{0111100}+\ket{1101001}+\ket{0000000}+\ket{1010101}
+\ket{0110011}+\ket{1100110})=\ket{0_L}$\\
Similarly, for $g_2$ and $g_3$.\\
For $g_3$ to $g_6$, each block has an even number of phase flips, hence
overall no overall phase flip takes place.\\
Similarly as above for the $\ket{1_L}$.
\subsection*{Exercise 10.33}
Let $r(g)=[\vec{x}|\vec{z}]$ and
$r(g^\prime)=[\vec{x}^\prime|\vec{z}^\prime]$. Then, \\
$r(g)\Lambda r(g^\prime)^T=\vec{x}.\vec{z^\prime}+
\vec{z}.\vec{x^\prime}$\\
If $g$ and $g^\prime$ commute then in total there are even number of
anti-commuting Pauli operators, hence the sum of the $2$ scalar products mod $2$
will be $0$. If $r(g)\Lambda r(g^\prime)^T=0$ then both scalar products will have to be $0$ or
$1$, hence there are an even number of anti-commuting Pauli operators, hence
$g$ and $g^\prime$ commute.
\subsection*{Exercise 10.34}
A counterexample is $S=<X, Z>$. $XZXZ=(-iY)(-iY)=-I$. 
\subsection*{Exercise 10.35}
Each $g$ is a tensor product of Pauli operators with prefactors
$\pm i$ or $\pm 1$, hence 
$g^2=\pm I$. However, $g^2\in S$, but $-I\notin S$, therefore
$g^2=I$.
\subsection*{Exercise 10.36}
$UX_2U^\dagger=
\begin{bmatrix}
I&0\\
0&X    
\end{bmatrix}
\begin{bmatrix}
    X&0\\
    0&X    
\end{bmatrix}
\begin{bmatrix}
    I&0\\
    0&X    
\end{bmatrix}=
\begin{bmatrix}
    I&0\\
    0&X    
\end{bmatrix}
\begin{bmatrix}
    X&0\\
    0&I    
\end{bmatrix}=
\begin{bmatrix}
    X&0\\
    0&X    
\end{bmatrix}=X_2$\\
$UZ_1U^\dagger=\begin{bmatrix}
    I&0\\
    0&X    
    \end{bmatrix}
    \begin{bmatrix}
        I&0\\
        0&-I    
    \end{bmatrix}
    \begin{bmatrix}
        I&0\\
        0&X    
    \end{bmatrix}=
    \begin{bmatrix}
        I&0\\
        0&X    
    \end{bmatrix}
    \begin{bmatrix}
        I&0\\
        0&-X    
    \end{bmatrix}=
    \begin{bmatrix}
        I&0\\
        0&-I    
    \end{bmatrix}=Z_1$\\
$UZ_2U^\dagger=\begin{bmatrix}
    I&0\\
    0&X    
    \end{bmatrix}
    \begin{bmatrix}
        Z&0\\
        0&Z    
    \end{bmatrix}
    \begin{bmatrix}
        I&0\\
        0&X    
    \end{bmatrix}=
    \begin{bmatrix}
        I&0\\
        0&X    
    \end{bmatrix}
    \begin{bmatrix}
        Z&0\\
        0&-iY    
    \end{bmatrix}=
    \begin{bmatrix}
        Z&0\\
        0&-Z    
    \end{bmatrix}=Z_1Z_2$
\subsection*{Exercise 10.37}
$UY_1U^\dagger=
\begin{bmatrix}
    I&0\\
    0&X    
    \end{bmatrix}
    \begin{bmatrix}
        0&-iI\\
        iI&0    
    \end{bmatrix}
    \begin{bmatrix}
        I&0\\
        0&X    
    \end{bmatrix}=
    \begin{bmatrix}
        I&0\\
        0&X    
    \end{bmatrix}
    \begin{bmatrix}
        0&-iX\\
        iI&0    
    \end{bmatrix}=
    \begin{bmatrix}
        0&-iX\\
        iX&0    
    \end{bmatrix}=Y_1X_2$
\subsection*{Exercise 10.38}
\subsection*{Exercise 10.39}
$SXS^\dagger=
\begin{bmatrix}
    1&0\\
    0&i
\end{bmatrix}
\begin{bmatrix}
    0&1\\
    1&0
\end{bmatrix}
\begin{bmatrix}
    1&0\\
    0&-i
\end{bmatrix}=
\begin{bmatrix}
    1&0\\
    0&i
\end{bmatrix}
\begin{bmatrix}
    0&-i\\
    1&0
\end{bmatrix}=
\begin{bmatrix}
    0&-i\\
    i&0
\end{bmatrix}=Y$\\
$SXS^\dagger=
\begin{bmatrix}
    1&0\\
    0&i
\end{bmatrix}
\begin{bmatrix}
    1&0\\
    0&-1
\end{bmatrix}
\begin{bmatrix}
    1&0\\
    0&-i
\end{bmatrix}=
\begin{bmatrix}
    1&0\\
    0&i
\end{bmatrix}
\begin{bmatrix}
    1&0\\
    0&i
\end{bmatrix}=
\begin{bmatrix}
    1&0\\
    1&-1
\end{bmatrix}=Z$
\subsection*{Exercise 10.40}
1) First, consider $UZU^\dagger=Z$, for this to be true
we require $U=\begin{bmatrix}
    1&0\\
    0&e^{i\phi}
\end{bmatrix}$. For this $U$ we see that, 
$UXU^\dagger=\pm X, \pm Y$, with $e^{i\phi}=\pm 1, \pm i$.
Therefore, we see that $U$ can be constructed using only phase gates.\\
From Chapter 4 we know that for and Pauli operator $\sigma$ there exists
$R$ constructed from Hadamards and phase gates, such that
$R\sigma R^\dagger =Z$.\\
Let's consider a normalizer $U$ for $G_1$. Then, $\exists g\in G_1$ such
that $UgU^\dagger=Z$. Let $U=VR$, where $R$ is defined as above.
Then, $UgU^\dagger=VRgR^\dagger V^\dagger=VZV^\dagger=Z$, hence from above
$V$ consists of only phase gates and $R$ consists of phase and Hadamard gates, therefore
$U$ consists of only phase and Hadamard gates.\\
Therefore, phase and Hadamard gates can be used to construct any normalizer one
$G_1$.\\
2) Let the process described by the circuit be $\bar{U}$. We like to show
$\bra{a}\bar{U}\ket{b}\ket{\psi}=\bra{a}U\ket{b}\ket{\psi}$ $\forall a,b,\psi$.\\
First we get the following from the conditions on $U$,\\
$UZ_1=(X_1\otimes g)U$\\
$X_1U=(I\otimes g)UZ_1=gUZ_1$\\
$UX_1=(Z_1\otimes g^\prime)U$\\
$Z_1U=(I\otimes g^\prime)UX_1=g^\prime UX_1$\\
Now consider $U^\prime\ket{\psi}$\\
$U^\prime\ket{\psi}=\sqrt{2}\bra{0}U(\ket{0}\ket{\psi})=
\sqrt{2}\bra{0}X_1gUZ_1(\ket{0}\ket{\psi})=
\sqrt{2}\bra{1}gU(\ket{0}\ket{\psi})$\\
$U^\prime\ket{\psi}=\sqrt{2}\bra{0}Z_1g^\prime UX_1(\ket{0}\ket{\psi})=
\sqrt{2}\bra{0}g^\prime U(\ket{1}\ket{\psi})$\\
$U^\prime\ket{\psi}=\sqrt{2}\bra{1}Z_1g g^\prime UX_1(\ket{0}\ket{\psi})
-\sqrt{2}\bra{1}g g^\prime U(\ket{1}\ket{\psi})$\\
Now consider, $\bra{a}\bar{U}\ket{b}\ket{\psi}$.\\
$\bra{0}\bar{U}\ket{0}\ket{\psi}=\bra{0}\frac{1}{\sqrt{2}}
(\ket{0}\otimes U^\prime \ket{\psi}+\ket{1}\otimes gU^\prime\ket{\psi})=
\frac{1}{\sqrt{2}}U^\prime\ket{\psi}=\bra{0}U\ket{0}\ket{\psi}$\\
$\bra{0}\bar{U}\ket{1}\ket{\psi}=\bra{0}\frac{1}{\sqrt{2}}
(\ket{0}\otimes g^\prime U^\prime \ket{\psi}-\ket{1}\otimes gg^\prime U^\prime\ket{\psi})=
\frac{1}{\sqrt{2}}g^\prime U^\prime\ket{\psi}=\bra{0}U\ket{1}\ket{\psi}$\\
$\bra{1}\bar{U}\ket{1}\ket{\psi}=\bra{1}\frac{1}{\sqrt{2}}
(\ket{0}\otimes g^\prime U^\prime \ket{\psi}-\ket{1}\otimes gg^\prime U^\prime\ket{\psi})=
-\frac{1}{\sqrt{2}}gg^\prime U^\prime\ket{\psi}=-\bra{1}U\ket{1}\ket{\psi}$\\
$\bra{1}\bar{U}\ket{0}\ket{\psi}=\bra{1}\frac{1}{\sqrt{2}}
(\ket{0}\otimes U^\prime \ket{\psi}+\ket{1}\otimes gU^\prime\ket{\psi})=
\frac{1}{\sqrt{2}}gU^\prime\ket{\psi}=\bra{1}U\ket{0}\ket{\psi}$\\
Hence, $\bra{a}\bar{U}\ket{b}\ket{\psi}=\bra{a}U\ket{b}\ket{\psi}$ $\forall a,b,\psi$, therefore
$U=\bar{U}$.\\
Overall, $U$ is composed of $U^\prime$ and $O(n)$ phase and Hadamard
gates. As construction of a gate $U\in N(G_{n+1})$ requires a gate
$U^\prime \in N(G_n)$, for gate $U$ we need $\displaystyle\sum_{i=1}^nO(i)=O(n^2)$ phase and
Hadamard gates.\\
3) Consider $UZ_1U^\dagger = g$ and $UX_1U^\dagger=g^\prime$. Then 
$\{g,g^\dagger\}=0$ as $\{Z_1,X_1\}=0$. Hence, $g$ and $g^\prime$ have at
some position $j$ $\sigma_j\neq\sigma_j^\prime$. Hence, we use the SWAP operator to turn the
situation of that of part (2).\\
$\textbf{SWAP}_{1j}UZ_1U^\dagger\textbf{SWAP}_{1j}^\dagger=\sigma\otimes g_1$\\
$\textbf{SWAP}_{1j}UX_1U^\dagger\textbf{SWAP}_{1j}^\dagger=\sigma^\prime\otimes g_1^\prime$\\
As we can construct pauli operators using Hadamard and phase gates,
if $\sigma\neq\sigma^\prime$ then $R\sigma R^\dagger=Z_1$ and $R\sigma^\prime R^\dagger=X_1$
for some $R$ constructed from phase and Hadamard gates. Then,\\
$R\textbf{SWAP}_{1j}UZ_1U^\dagger\textbf{SWAP}_{1j}^\dagger R^\dagger=Z_1\otimes g_1$\\
$R\textbf{SWAP}_{1j}UZ_1U^\dagger\textbf{SWAP}_{1j}^\dagger R^\dagger=X_1\otimes g_1$\\
which is the situation of part (2).\\
Therefore, as the \textbf{SWAP} is made out of 3 \textbf{CNOT}s, we conclude
that any normalizer can be written as a composition of $O(n^2)$ phase, Hadamard and
\textbf{CNOT} gates.
\subsection*{Exercise 10.41}
$T=\begin{bmatrix}
    1&0\\
    0&e^{i\pi/4}
\end{bmatrix}$\\
$TZT^\dagger=\begin{bmatrix}
    1&0\\
    0&e^{i\pi/4}
\end{bmatrix}
\begin{bmatrix}
    1&0\\
    0&-1
\end{bmatrix}
\begin{bmatrix}
    1&0\\
    0&e^{-i\pi/4}
\end{bmatrix}=
\begin{bmatrix}
    1&0\\
    0&-e^{i\pi/4}e^{-i\pi/4}
\end{bmatrix}=Z$\\
$TXT^\dagger=\begin{bmatrix}
    1&0\\
    0&e^{i\pi/4}
\end{bmatrix}
\begin{bmatrix}
    0&1\\
    1&0
\end{bmatrix}
\begin{bmatrix}
    1&0\\
    0&e^{-i\pi/4}
\end{bmatrix}=
\begin{bmatrix}
    1&0\\
    0&e^{i\pi/4}
\end{bmatrix}
\begin{bmatrix}
    0&e^{-i\pi/4}\\
    1&0
\end{bmatrix}=
\begin{bmatrix}
    0&e^{-i\pi/4}\\
    e^{i\pi/4}&0
\end{bmatrix}=
\begin{bmatrix}
    0&\frac{1-i}{\sqrt{2}}\\
    \frac{1+i}{\sqrt{2}}&0
\end{bmatrix}=\frac{X+Y}{\sqrt{2}}$\\
$U=\begin{bmatrix}
    I&0&0&0\\
    0&I&0&0\\
    0&0&I&0\\
    0&0&0&X
\end{bmatrix}$\\
$UZ_1U^\dagger=\begin{bmatrix}
    I&0&0&0\\
    0&I&0&0\\
    0&0&I&0\\
    0&0&0&X
\end{bmatrix}
\begin{bmatrix}
    I&0&0&0\\
    0&I&0&0\\
    0&0&-I&0\\
    0&0&0&-I
\end{bmatrix}
\begin{bmatrix}
    I&0&0&0\\
    0&I&0&0\\
    0&0&I&0\\
    0&0&0&X
\end{bmatrix}=
\begin{bmatrix}
    I&0&0&0\\
    0&I&0&0\\
    0&0&-I&0\\
    0&0&0&-XX
\end{bmatrix}=Z_1$
$UZ_2U^\dagger=\begin{bmatrix}
    I&0&0&0\\
    0&I&0&0\\
    0&0&I&0\\
    0&0&0&X
\end{bmatrix}
\begin{bmatrix}
    I&0&0&0\\
    0&-I&0&0\\
    0&0&I&0\\
    0&0&0&-I
\end{bmatrix}
\begin{bmatrix}
    I&0&0&0\\
    0&I&0&0\\
    0&0&I&0\\
    0&0&0&X
\end{bmatrix}=
\begin{bmatrix}
    I&0&0&0\\
    0&-I&0&0\\
    0&0&I&0\\
    0&0&0&-XX
\end{bmatrix}=Z_2$
$UX_3U^\dagger=\begin{bmatrix}
    I&0&0&0\\
    0&I&0&0\\
    0&0&I&0\\
    0&0&0&X
\end{bmatrix}
\begin{bmatrix}
    X&0&0&0\\
    0&X&0&0\\
    0&0&X&0\\
    0&0&0&X
\end{bmatrix}
\begin{bmatrix}
    I&0&0&0\\
    0&I&0&0\\
    0&0&I&0\\
    0&0&0&X
\end{bmatrix}=
\begin{bmatrix}
    X&0&0&0\\
    0&X&0&0\\
    0&0&X&0\\
    0&0&0&XXX
\end{bmatrix}=X_3$\\
$UX_1U^\dagger=\begin{bmatrix}
    I&0&0&0\\
    0&I&0&0\\
    0&0&I&0\\
    0&0&0&X
\end{bmatrix}
\begin{bmatrix}
    0&0&I&0\\
    0&0&0&I\\
    I&0&0&0\\
    0&I&0&0
\end{bmatrix}
\begin{bmatrix}
    I&0&0&0\\
    0&I&0&0\\
    0&0&I&0\\
    0&0&0&X
\end{bmatrix}=
\begin{bmatrix}
    I&0&0&0\\
    0&I&0&0\\
    0&0&I&0\\
    0&0&0&X
\end{bmatrix}
\begin{bmatrix}
    0&0&I&0\\
    0&0&0&X\\
    I&0&0&0\\
    0&I&0&0
\end{bmatrix}=
\begin{bmatrix}
    0&0&I&0\\
    0&0&0&X\\
    I&0&0&0\\
    0&X&0&0
\end{bmatrix}=X_1\otimes
\begin{bmatrix}
    I&0\\
    0&X
\end{bmatrix}=X_1\otimes
\frac{
    \begin{bmatrix}
        I&0\\
        0&I
    \end{bmatrix}+
    \begin{bmatrix}
        I&0\\
        0&-I
    \end{bmatrix}+
    \begin{bmatrix}
        X&0\\
        0&X
    \end{bmatrix}+
    \begin{bmatrix}
        -X&0\\
        0&X
    \end{bmatrix}
}{2}=\\
\displaystyle X_1\otimes\frac{I+Z_2+X_3-Z_2X_3}{2}$\\
$UX_2U^\dagger=\begin{bmatrix}
    I&0&0&0\\
    0&I&0&0\\
    0&0&I&0\\
    0&0&0&X
\end{bmatrix}
\begin{bmatrix}
    0&I&0&0\\
    I&0&0&0\\
    0&0&0&I\\
    0&0&I&0
\end{bmatrix}
\begin{bmatrix}
    I&0&0&0\\
    0&I&0&0\\
    0&0&I&0\\
    0&0&0&X
\end{bmatrix}=
\begin{bmatrix}
    I&0&0&0\\
    0&I&0&0\\
    0&0&I&0\\
    0&0&0&X
\end{bmatrix}
\begin{bmatrix}
    0&I&0&0\\
    I&0&0&0\\
    0&0&0&X\\
    0&0&I&0
\end{bmatrix}=
\begin{bmatrix}
    0&I&0&0\\
    I&0&0&0\\
    0&0&0&X\\
    0&0&X&0
\end{bmatrix}=\frac{1}{2}\left\{
\begin{bmatrix}
    0&I&0&0\\
    I&0&0&0\\
    0&0&0&I\\
    0&0&I&0
\end{bmatrix}+
\begin{bmatrix}
    0&I&0&0\\
    I&0&0&0\\
    0&0&0&-I\\
    0&0&-I&0
\end{bmatrix}+
\begin{bmatrix}
    0&X&0&0\\
    X&0&0&0\\
    0&0&0&X\\
    0&0&X&0
\end{bmatrix}+
\begin{bmatrix}
    0&-X&0&0\\
    -X&0&0&0\\
    0&0&0&X\\
    0&0&X&0
\end{bmatrix}
\right\}\\
=\displaystyle X_2\otimes\frac{I+Z_1+X_3-Z_1X_3}{2}$\\
$UZ_3U^\dagger=\begin{bmatrix}
    I&0&0&0\\
    0&I&0&0\\
    0&0&I&0\\
    0&0&0&X
\end{bmatrix}
\begin{bmatrix}
    Z&0&0&0\\
    0&Z&0&0\\
    0&0&Z&0\\
    0&0&0&Z
\end{bmatrix}
\begin{bmatrix}
    I&0&0&0\\
    0&I&0&0\\
    0&0&I&0\\
    0&0&0&X
\end{bmatrix}=
\begin{bmatrix}
    Z&0&0&0\\
    0&Z&0&0\\
    0&0&Z&0\\
    0&0&0&XZX
\end{bmatrix}=
\begin{bmatrix}
    Z&0&0&0\\
    0&Z&0&0\\
    0&0&Z&0\\
    0&0&0&-Z
\end{bmatrix}=\frac{1}{2}\left\{
    \begin{bmatrix}
        Z&0&0&0\\
        0&Z&0&0\\
        0&0&Z&0\\
        0&0&0&Z
    \end{bmatrix}+
    \begin{bmatrix}
        Z&0&0&0\\
        0&Z&0&0\\
        0&0&-Z&0\\
        0&0&0&-Z
    \end{bmatrix}+
    \begin{bmatrix}
        Z&0&0&0\\
        0&-Z&0&0\\
        0&0&Z&0\\
        0&0&0&-Z
    \end{bmatrix}+
    \begin{bmatrix}
        -Z&0&0&0\\
        0&Z&0&0\\
        0&0&Z&0\\
        0&0&0&-Z
    \end{bmatrix}
\right\}\\
=\displaystyle Z_3\otimes\frac{I+Z_1+Z_2-Z_1Z_2}{2}$
\subsection*{Exercise 10.42}
Initially $S=<IXX,IZZ>$ with $\bar{Z}=ZII$ and $\bar{X}=XII$.
Considering the effect of the circuit on the generators we get,\\
$IXX\xrightarrow{CNOT}IXX\xrightarrow{H}IXX
\xrightarrow{\text{Mes. }X_1}IXX
\xrightarrow{\text{Mes. }Z_2}IZI$\\
$IZZ\xrightarrow{CNOT}ZZZ\xrightarrow{H}XZZ
\xrightarrow{\text{Mes. }X_1}XZZ
\xrightarrow{\text{Mes. }Z_2}XZZ$\\
For the final $S_f=<IZI,XZZ>$ we have $\bar{Z}=IIZ$ and $\bar{X}=IIX$,
hence the circuit does indeed teleport the initial state.
\subsection*{Exercise 10.43}
$\forall g \in S$ we have $g\in N(S)$ as $gg^\prime g^\dagger \in S$ $\forall g^\prime\in S$ due to
$S$ being a group. Therefore, $S\subseteq N(S)$.
\subsection*{Exercise 10.44}
\subsection*{Exercise 10.45}

\subsection*{Exercise 10.46}
$S=<X_1X_2,X_2X_3>=\{I,X_1X_2,X_2X_3,X_1X_3\}$\\
The subspace fixed by $X_1X_2$ is spanned by $\ket{+++},\ket{++-},\ket{---}$ and $\ket{--+}$.
The subspace fixed by $X_2X_3$ is spanned by $\ket{+++},\ket{-++},\ket{---}$ and $\ket{-++}$.
Hence, the subspace fixed by $S$ is spanned by $\ket{+++}$ and $\ket{---}$, which is the subspace
for the three qubit phase flip code. Therefore, $X_1X_2$ and $X_2X_3$ generate the stabilizer
for the three qubit phase flip code. 
\subsection*{Exercise 10.47}
The generators 1-6 have 2 $Z$'s with both $Z$'s being in one of the triplets,
hence the phase flips are cancelled and $\ket{0_L}$ and $\ket{1_L}$ are fixed by them.\\
Generators 7 and 8 have $X$s on all elements of any of the triplets or none. Hence, they fix
$\ket{0_L}$. As 2 triplets are acted on, the phase flip from the triplets is cancelled and 
hence $\ket{1_L}$ is also fixed. Therefore, theses are the generators for the Shor-code.
\subsection*{Exercise 10.48}
Each generator has an even number of $Z$'s or $X$'s, hence commute with $\bar{Z}$ and $\bar{X}$.
Counting $Y$'s as both an $X$ and a $Z$, any product of the generators has an even number of $Z$'s
and an even number of $X$'s, therefore as $\bar{Z}$ has an odd number of $Z$'s and
$\bar{X}$ has an odd number of $X$'s they are independent of the generators. Lastly,
$\bar{X}\bar{Z}=(XZ)^{\otimes 9}=(-1)^9(ZX)^{\otimes 9}=-\bar{Z}\bar{X}$. Therefore,
$\bar{Z}$ and $\bar{X}$ act as logical $Z$ and $X$ operators for the Shor-code.
\subsection*{Exercise 10.49}
Consider the set $E=\{X_1,\ldots X_5,Y_1\ldots Y_5, Z_1,\ldots Z_5\}$.
Consider for example the combination $X_1Z_2$. It commutes with $g_2$ hence $X_1Z_2\notin N(S)$.
Similarly, we can show that $\forall E_iE_j$, $E_iE_j\notin N(S)$ or $E_iE_j\in S$, and hence
$E_iE_j\notin N(S)-S$, therefore by Theorem 10.8 the five qubit code can protect against an
arbitrary single qubit error. 
\subsection*{Exercise 10.50}
For the five qubit code $t=1$, $n=5$ and $k=1$. Hence, the Hamming bound is\\
$\displaystyle{5\choose 0}3^02^1+{5\choose 1}3^12^1=2+5\times 6=32=2^5$\\
Therefore, the five qubit code saturates the Hamming bound.
\subsection*{Exercise 10.51}
The given check matrix is split into generators with only $X$'s and generators only into $Z$'s.
Consider the set $E$ of all possible $t$ operator tensor products of $X$'s and $Z$'s. Consider
$E_iE_j$. As both $C_1$ and $C_2$ correct up to $t$ errors, $\exists g\in S$ such that 
for the $2t$ length row $E_iE_j$ $E_iE_jgE_iE_j\notin S$ or $E_iE_j\in S$. Hence,
$E_iE_j\notin N(S)-S$, therefore, by Theorem 10.8 $E$ is a length $t$ set of correctable errors.
\subsection*{Exercise 10.52}
Using figure 10.6 for the stabilizers. Each generator has an even number of $Z$'s or
$X$'s, hence commute with $\bar{Z}$ and $\bar{X}$. 
Counting $Y$'s as both an $X$ and a $Z$, any product of the generators has an even number of $Z$'s
and an even number of $X$'s, therefore as $\bar{Z}$ has an odd number of $Z$'s and
$\bar{X}$ has an odd number of $X$'s they are independent of the generators. Lastly,
$\bar{X}\bar{Z}=(XZ)^{\otimes 7}=(-1)^7(ZX)^{\otimes 7}=-\bar{Z}\bar{X}$. Therefore,
$\bar{Z}$ and $\bar{X}$ act as logical $Z$ and $X$ operators for the Steane-code.
\subsection*{Exercise 10.53}
As $G_z$ includes $I$, the rank of $G_z$ is $k$, hence all of its rows are independent and
therefore the encoded $Z$ operators are independent of each other.
\subsection*{Exercise 10.54}
$G_x=[0E^TI|C^T00]$\\
For the first $r$ generators we have $I\times (C^T)^T+C\times I=0$ and for
the other $n-k-r$ generators $I\times (E^T)^T+E\times I=0$, therefore the encoded $X$
operators commute with the generators. The encoded $X$ operators commute with each other,
as in each block we have only $X$ or only $Z$ operators. The independence from the generators
follows from the $I$ in the left corner of the check matrix, and the $0$s in the last
$n-k-r$ generators, as it's not possible to get the $X$ section of $G_x$ with these, due to
the $0$ in the first block in $G_x$. As $G_x$ includes $I$, the rank of $G_x$ is $k$, 
hence all of its rows are independent and
therefore the encoded $X$ operators are independent of each other.
Comparing $G_x$ and $G_z$, in the first two blocks we have only $X$'s or only $Z$'s, hence
those parts commute. The third block for both the $X$ and $Z$ section is the identity, hence for
$\bar{X}_i$ the third block corresponds with the $i^{th}$ row of $I$ in the $X$ section and
similarly for $\bar{Z}_j$ to the $j^{th}$ row of $I$ in the $Z$ section. Hence, $\bar{X}_i$
and $\bar{Z}_j$ have an $X$ and a $Z$ respectively in the same location only if $i=j$. Therefore,
$\bar{X}_i$ and $\bar{Z}_j$ commute unless $i=j$ in which case they anti-commute.
\subsection*{Exercise 10.55}
$E=(1,1,0)$ and $C=(0,0,0)$, $\bar{X}=X_4X_5X_7$.
\subsection*{Exercise 10.56}
Consider $g_i\bar{X}_i$, as $g_i$ commutes with and is independent of all the 
$g$'s and $\bar{X}$'s and $\bar{Z}$'s,
$g_i\bar{X}_i$ still commutes with and is independent of all of them. Also, from the previous
statement $g_i\bar{X}_i$ commutes with all $\bar{Z}_j$ unless $i=j$. Similarly, for 
$g_i\bar{Z}_i$. Therefore, $gX$ and $gZ$ still are encoded $X$ and $Z$ operators and hence
their action on the code doesn't change.
\subsection*{Exercise 10.57}
For the five qubit code swap $q_2$ and $q_5$, replace $g_1$ by $g_1g_4$ and afterwards
$g_3$ with $g_3g_1$.\\
$G_4=\left[\begin{matrix}
    1&0&0&1&0\\
    0&1&0&0&1\\
    1&0&1&0&0\\
    0&1&0&1&0
\end{matrix}
\middle\vert
\begin{matrix}
    0&1&1&0&0\\
    0&0&1&1&0\\
    0&0&0&1&1\\
    1&0&0&0&1
\end{matrix}\right]
=\left[\begin{matrix}
    1&0&0&0&1\\
    0&1&0&0&1\\
    0&0&1&0&1\\
    0&0&0&1&1
\end{matrix}
\middle\vert
\begin{matrix}
    1&1&1&0&1\\
    0&0&1&1&0\\
    1&0&1&1&1\\
    1&1&0&0&0
\end{matrix}\right]$\\
For the nine qubit code replace $g_3$ by $g_3g_4$, $g_5$ by $g_5g_6$,
$g_7$ by $g_7g_8$. Then swap $q_1$ and $q_3$, $q_2$ and $q_4$,
$q_2$ and $q_5$, $q_2$ and $q_6$, $q_2$ and $q_8$\\
$G_9=\left[\begin{matrix}
    1&1&1&1&1&1&0&0&0\\
    0&0&0&1&1&1&1&1&1\\
    0&0&0&0&0&0&0&0&0\\
    0&0&0&0&0&0&0&0&0\\
    0&0&0&0&0&0&0&0&0\\
    0&0&0&0&0&0&0&0&0\\
    0&0&0&0&0&0&0&0&0\\
    0&0&0&0&0&0&0&0&0
\end{matrix}
\middle\vert
\begin{matrix}
    0&0&0&0&0&0&0&0&0\\
    0&0&0&0&0&0&0&0&0\\
    1&1&0&0&0&0&0&0&0\\
    0&1&1&0&0&0&0&0&0\\
    0&0&0&1&1&0&0&0&0\\
    0&0&0&0&1&1&0&0&0\\
    0&0&0&0&0&0&1&1&0\\
    0&0&0&0&0&0&0&1&1
\end{matrix}\right]=\\
\\
\left[\begin{matrix}
    1&0&1&1&1&1&0&0&1\\
    0&1&0&0&1&1&1&1&1\\
    0&0&0&0&0&0&0&0&0\\
    0&0&0&0&0&0&0&0&0\\
    0&0&0&0&0&0&0&0&0\\
    0&0&0&0&0&0&0&0&0\\
    0&0&0&0&0&0&0&0&0\\
    0&0&0&0&0&0&0&0&0
\end{matrix}
\middle\vert
\begin{matrix}
    0&0&0&0&0&0&0&0&0\\
    0&0&0&0&0&0&0&0&0\\
    1&0&1&0&0&0&0&0&0\\
    1&0&0&1&0&0&0&0&0\\
    0&0&0&0&1&0&0&0&1\\
    0&0&0&0&0&1&0&0&1\\
    0&1&0&0&0&0&1&0&0\\
    0&1&0&0&0&0&0&1&0    
\end{matrix}\right]$
\subsection*{Exercise 10.58}
From Exercise 4.34 the circuits function as described. The equivalence of Figure 10.14
follows from Exercise 4.20 and the equivalence of Figure 10.15 follows from Exercise 4.18 and
that $HZH=X$.
\subsection*{Exercise 10.59}
Each single control multiple targets gate can be written as multiple
single control single target gates. As $HH=I$ we can add Hadamard pairs between
the single control single target gates, which then gives multiple subcircuits in the 
form of Figures 10.14 and 15. Applying, the equivalence of the circuits using $HH=I$ 
and collecting the single control single target gates together under single target
multiple targets gates gives the circuit in Figure 10.17.
\newpage
\subsection*{Exercise 10.60}

Using the check matrices from Exercise 10.57 we get,\\
For the five qubit code,\\
$\begin{quantikz}
    \lstick{$\ket{0}$}&\gate{H}&\ctrl{8}&\qw&\qw&\qw&\ctrl{8}&\qw&\qw&\qw&\gate{H}&\qw&\meter{}\\
    \lstick{$\ket{0}$}&\gate{H}&\qw&\ctrl{7}&\qw&\qw&\qw&\ctrl{6}&\qw&\qw&\gate{H}&\qw&\meter{}\\
    \lstick{$\ket{0}$}&\gate{H}&\qw&\qw&\ctrl{6}&\qw&\qw&\qw&\ctrl{6}&\qw&\gate{H}&\qw&\meter{}\\
    \lstick{$\ket{0}$}&\gate{H}&\qw&\qw&\qw&\ctrl{5}&\qw&\qw&\qw&\ctrl{2}&\gate{H}&\qw&\meter{}\\
    &\qw&\gate{X}&\qw&\qw&\qw&\gate{Z}&\qw&\gate{Z}&\gate{Z}&\qw&\qw\\
    &\qw&\qw&\gate{X}&\qw&\qw&\gate{Z}&\qw&\qw&\gate{Z}&\qw&\qw\\
    &\qw&\qw&\qw&\gate{X}&\qw&\gate{Z}&\gate{Z}&\gate{Z}&\qw&\qw&\qw\\
    &\qw&\qw&\qw&\qw&\gate{X}&\qw&\gate{Z}&\gate{Z}&\qw&\qw&\qw\\
    &\qw&\gate{X}&\gate{X}&\gate{X}&\gate{X}&\gate{Z}&\qw&\gate{Z}&\qw&\qw&\qw\\
\end{quantikz}$\\
\newpage

For the nine qubit code,\\
$\begin{quantikz}
    \lstick{$\ket{0}$}&\gate{H}&\ctrl{13}&\qw&\qw&\qw&\qw&\qw&\qw&\qw&\gate{H}&\qw&\meter{}\\
    \lstick{$\ket{0}$}&\gate{H}&\qw&\ctrl{15}&\qw&\qw&\qw&\qw&\qw&\qw&\gate{H}&\qw&\meter{}\\
    \lstick{$\ket{0}$}&\gate{H}&\qw&\qw&\ctrl{12}&\qw&\qw&\qw&\qw&\qw&\gate{H}&\qw&\meter{}\\
    \lstick{$\ket{0}$}&\gate{H}&\qw&\qw&\qw&\ctrl{11}&\qw&\qw&\qw&\qw&\gate{H}&\qw&\meter{}\\
    \lstick{$\ket{0}$}&\gate{H}&\qw&\qw&\qw&\qw&\ctrl{8}&\qw&\qw&\qw&\gate{H}&\qw&\meter{}\\
    \lstick{$\ket{0}$}&\gate{H}&\qw&\qw&\qw&\qw&\qw&\ctrl{8}&\qw&\qw&\gate{H}&\qw&\meter{}\\
    \lstick{$\ket{0}$}&\gate{H}&\qw&\qw&\qw&\qw&\qw&\qw&\ctrl{9}&\qw&\gate{H}&\qw&\meter{}\\
    \lstick{$\ket{0}$}&\gate{H}&\qw&\qw&\qw&\qw&\qw&\qw&\qw&\ctrl{9}&\gate{H}&\qw&\meter{}\\
    &\qw&\gate{X}&\qw&\gate{Z}&\qw&\qw&\qw&\qw&\qw&\qw&\qw\\
    &\qw&\qw&\gate{X}&\qw&\qw&\qw&\qw&\gate{Z}&\qw&\qw&\qw\\
    &\qw&\gate{X}&\qw&\qw&\gate{Z}&\qw&\qw&\qw&\qw&\qw&\qw\\
    &\qw&\gate{X}&\gate{X}&\qw&\qw&\gate{Z}&\qw&\qw&\qw&\qw&\qw\\
    &\qw&\gate{X}&\gate{X}&\qw&\qw&\gate{Z}&\gate{Z}&\qw&\qw&\qw&\qw\\
    &\qw&\gate{X}&\gate{X}&\qw&\qw&\qw&\gate{Z}&\qw&\qw&\qw&\qw\\
    &\qw&\gate{X}&\qw&\gate{Z}&\gate{Z}&\qw&\qw&\qw&\qw&\qw&\qw\\
    &\qw&\qw&\gate{X}&\qw&\qw&\qw&\qw&\gate{Z}&\gate{Z}&\qw&\qw\\
    &\qw&\qw&\gate{X}&\qw&\qw&\qw&\qw&\qw&\gate{Z}&\qw&\qw
\end{quantikz}$
\subsection*{Exercise 10.61}
$E_j$ are all the possible single qubit errors. For each $E_j$ we consider,
$E_jg_lE_j^\dagger=\beta_lg_l$ where $\beta_l$ is the syndrome measurement. Hence, for 
each syndrome we compare the $\beta_l$'s and decide based on that which $E_j$ has occurred
and apply $E_j^\dagger$ to recover.
We also have seen that if multiple $E_j$ give rise to the same syndrome then it suffices to
choose one of them and applying $E_j^\dagger$ for that error operator corrects all
the errors with the same syndrome.
\subsection*{Exercise 10.62}
Let $S_1$ be the generator for $[n_1,1]$ and $S_2$ the generator for $[n_2,1]$.
After concatenating the generators for the stabilizer will need to stabilize both $[n_1,1]$
and $[n_2,1]$. 
Hence, the generators will be all the combinations $g_ig_j^\prime$ where $g_i\in S_1$ and
$g_j^\prime\in S_2$. Therefore, this will create a $[n_1n_2,1]$ code.
\subsection*{Exercise 10.63}
$\bar{U}\ket{0_L}=\bar{U}\bar{Z}\ket{0_L}=\bar{X}\bar{U}\ket{0_L}$\\
$\bar{U}\ket{1_L}=-\bar{U}\bar{Z}\ket{1_L}=-\bar{X}\bar{U}\ket{1_L}$\\
Hence, $\bar{U}\ket{0_L}$ and $\bar{U}\ket{1_L}$ correspond to the $\pm 1 $ eigenstates
of $\bar{X}$, hence up to a phase factor $\bar{U}\ket{0_L}=\frac{1}{\sqrt{2}}(\ket{0_L}+\ket{1_L})$
and $\bar{U}\ket{1_L}=\frac{1}{\sqrt{2}}(\ket{0_L}-\ket{1_L})$.
\subsection*{Exercise 10.64}
Using Exercise 10.36 $UZ_2=UZ_2U^\dagger U=Z_1Z_2U$.\\
Using Exercise 4.20, $HZ=XH$ and the propagation of an $X$ error on the control qubit,\\
$\begin{quantikz}
    &\qw&\ctrl{1}&\qw\\
    &\gate{Z}&\targ{}&\qw
\end{quantikz}=
\begin{quantikz}
    &\qw&\gate{H}&\gate{H}&\ctrl{1}&\gate{H}&\gate{H}&\qw\\
    &\gate{Z}&\gate{H}&\gate{H}&\targ{}&\gate{H}&\gate{H}&\qw
\end{quantikz}=
\begin{quantikz}
    &\qw&\gate{H}&\targ{}&\gate{H}&\qw\\
    &\gate{H}&\gate{X}&\ctrl{-1}&\gate{H}&\qw
\end{quantikz}=\\
\begin{quantikz}
    &\qw&\gate{H}&\targ{}&\gate{X}&\gate{H}&\qw\\
    &\qw&\gate{H}&\ctrl{-1}&\gate{X}&\gate{H}&\qw
\end{quantikz}=
\begin{quantikz}
    &\qw&\gate{H}&\targ{}&\gate{H}&\gate{Z}&\qw\\
    &\qw&\gate{H}&\ctrl{-1}&\gate{H}&\gate{Z}&\qw
\end{quantikz}=
\begin{quantikz}
    &\qw&\ctrl{1}&\gate{Z}&\qw\\
    &\qw&\targ{}&\gate{Z}&\qw
\end{quantikz}$
\subsection*{Exercise 10.65}
Let $\ket{\psi}=a\ket{0}+b\ket{1}$. For the first circuit we have,\\
$a\ket{00}+b\ket{01}\xrightarrow{\text{CNOT}_{21}}a\ket{00}+b\ket{11}\xrightarrow{H_2}
\frac{1}{\sqrt{2}}(a\ket{00}+a\ket{01}+b\ket{10}-b\ket{11})=\\
\frac{1}{\sqrt{2}}((a\ket{0}+b\ket{1})\ket{0}+(a\ket{0}-b\ket{1})\ket{1})\xrightarrow{\text{Mes}}
\begin{cases}
    \ket{\psi} & \text{if measurement gives} +1\\
    Z(a\ket{0}-b\ket{1})=\ket{\psi} & \text{if measurement gives} -1
\end{cases}$\\
For the second circuit we have,\\
$a\ket{00}+b\ket{01}\xrightarrow{H_1}\frac{1}{\sqrt{2}}(\ket{0}(a\ket{0}+b\ket{1})+
\ket{1}(a\ket{0}+b\ket{1}))\xrightarrow{\text{CNOT}_{12}}\\
\frac{1}{\sqrt{2}}(\ket{0}(a\ket{0}+b\ket{1})+\ket{1}(a\ket{1}+b\ket{0}))=\\
\frac{1}{\sqrt{2}}((a\ket{0}+b\ket{1})\ket{0}+(a\ket{1}+b\ket{0})\ket{1})\xrightarrow{\text{Mes}}
\begin{cases}
    \ket{\psi} & \text{if measurement gives} +1\\
    X(b\ket{0}+a\ket{1})=\ket{\psi} & \text{if measurement gives} -1
\end{cases}$
\subsection*{Exercise 10.66}
Mistake in the question, should be $TXT^\dagger =exp(-i\pi/4)SX$,i.e. $TX=exp(-i\pi/4)SXT$.\\
$\begin{quantikz}
    \lstick{$\ket{0}$}&\gate{H}&\ctrl{1}&\gate{X}&\gate{T}&\qw\rstick{$T\ket{\psi}$}\\
    \lstick{$\ket{\psi}$}&\qw&\targ{}&\meter{}\vcw{-1}
\end{quantikz}=
\begin{quantikz}
    \lstick{$\ket{0}$}&\gate{H}&\ctrl{1}&\gate{T}&\gate{SX}&\qw\rstick{$T\ket{\psi}$}\\
    \lstick{$\ket{\psi}$}&\qw&\targ{}&\qw&\meter{}\vcw{-1}
\end{quantikz}=\\
\begin{quantikz}
    \lstick{$\ket{0}$}&\gate{H}&\gate{T}&\ctrl{1}&\gate{SX}&\qw\rstick{$T\ket{\psi}$}\\
    \lstick{$\ket{\psi}$}&\qw&\qw&\targ{}&\meter{}\vcw{-1}
\end{quantikz}$
\subsection*{Exercise 10.67}
a)$LHS=\begin{bmatrix}
    I&0&0&0\\
    0&I&0&0\\
    0&0&I&0\\
    0&0&0&X
\end{bmatrix}
\begin{bmatrix}
    0&0&I&0\\
    0&0&0&I\\
    I&0&0&0\\
    0&I&0&0
\end{bmatrix}=
\begin{bmatrix}
    0&0&I&0\\
    0&0&0&I\\
    I&0&0&0\\
    0&X&0&0
\end{bmatrix}
$\\
$RHS=
\begin{bmatrix}
    0&0&I&0\\
    0&0&0&X\\
    I&0&0&0\\
    0&X&0&0
\end{bmatrix}
\begin{bmatrix}
    I&0&0&0\\
    0&I&0&0\\
    0&0&I&0\\
    0&0&0&X
\end{bmatrix}=
\begin{bmatrix}
    0&0&I&0\\
    0&0&0&XX\\
    I&0&0&0\\
    0&X&0&0
\end{bmatrix}=
\begin{bmatrix}
    0&0&I&0\\
    0&0&0&I\\
    I&0&0&0\\
    0&X&0&0
\end{bmatrix}$\\
b)$LHS=\begin{bmatrix}
    I&0&0&0\\
    0&I&0&0\\
    0&0&I&0\\
    0&0&0&X
\end{bmatrix}
\begin{bmatrix}
    Z&0&0&0\\
    0&Z&0&0\\
    0&0&Z&0\\
    0&0&0&Z
\end{bmatrix}=
\begin{bmatrix}
    Z&0&0&0\\
    0&Z&0&0\\
    0&0&Z&0\\
    0&0&0&XZ
\end{bmatrix}
$\\
$RHS=
\begin{bmatrix}
    Z&0&0&0\\
    0&Z&0&0\\
    0&0&Z&0\\
    0&0&0&-Z
\end{bmatrix}
\begin{bmatrix}
    I&0&0&0\\
    0&I&0&0\\
    0&0&I&0\\
    0&0&0&X
\end{bmatrix}=
\begin{bmatrix}
    Z&0&0&0\\
    0&Z&0&0\\
    0&0&Z&0\\
    0&0&0&-ZX
\end{bmatrix}=
\begin{bmatrix}
    Z&0&0&0\\
    0&Z&0&0\\
    0&0&Z&0\\
    0&0&0&XZ
\end{bmatrix}$
\subsection*{Exercise 10.68}
\subsection*{Exercise 10.69}

\end{document}