\documentclass[a4paper,12pt]{article}
\usepackage{amsmath}
\usepackage[margin=0.9in]{geometry}
\usepackage{braket}
\usepackage{graphicx}
\begin{document}
\subsection*{Exercise 4.1}
The eigenvectors are as follows:\\
Pauli $Z$: $\ket{0}$, $\ket{1}$\\
Pauli $X$: $\ket{0}+\ket{1}$, $\ket{0}-\ket{1}$\\
Pauli $Y$: $\ket{0}+i\ket{1}$, $\ket{0}-i\ket{1}$\\
Bloch sphere representations:

\includegraphics*[scale=0.4]{4.1.png}
\subsection*{Exercise 4.2}
exp$(iAx)=\displaystyle\sum_n(iAx)^n=
\displaystyle\sum_n(-1)^nx^{2n}I+\displaystyle\sum_n(-1)^nix^nA=
\cos(x)I+i\sin{x}A$
\subsection*{Exercise 4.3}
Up to a global phase:\\
$T=\noindent\begin{bmatrix}
    e^{-i\pi/8} &0\\
    0 & e^{i\pi/8}
\end{bmatrix}=
\begin{bmatrix}
    e^{-i\frac{\pi}{4}/2} &0\\
    0 & e^{i\frac{\pi}{4}/2}
\end{bmatrix}=
R_z(\pi/4)$
\subsection*{Exercise 4.4}
First consider $R_zR_xR_z$:\\
$R_zR_xR_z=\begin{bmatrix}
    \cos{\frac{\theta}{2}}e^{-i\theta} & -i\sin{\frac{\theta}{2}}\\
    -i\sin{\frac{\theta}{2}} & \cos{\frac{\theta}{2}}e^{i\theta}
\end{bmatrix}$\\
For $\theta=\frac{\pi}{2}$:\\
$R_zR_xR_z=\frac{1}{\sqrt{2}}\begin{bmatrix}
  e^{-i\frac{\pi}{2}} &  e^{-i\frac{\pi}{2}} \\
  e^{-i\frac{\pi}{2}}  & e^{i\frac{\pi}{2}}
\end{bmatrix}$\\
Hence, by multiplying by $e^{i\frac{\pi}{2}}$ we get,\\
$e^{i\frac{\pi}{2}}R_zR_xR_z=\frac{1}{\sqrt{2}}\begin{bmatrix}
    1 &  1 \\
   1  & -1
  \end{bmatrix}=H$\\
\subsection*{Exercise 4.5}
We have $n_x^2+n_y^2+n_z^2=1$\\
$\hat{n}\cdot\vec{\sigma}=\begin{bmatrix}
    n_z& n_x-in_y\\
    n_x+in_y& n_z
\end{bmatrix}$\\
Therefore,\\
$(\hat{n}\cdot\vec{\sigma})^2=\begin{bmatrix}
    n_x^2+n_y^2+n_z^2&0\\
    0&n_x^2+n_y^2+n_z^2
\end{bmatrix}=\begin{bmatrix}
    1&0\\
    0&1
\end{bmatrix}=I$\\
Consider, $R_n(\theta)R_n(-\theta)$\\
$I=R_n(\theta)R_n(-\theta)=(\cos(\frac{\theta}{2})I-\sin(\frac{\theta}{2})\hat{n}\cdot\vec{\sigma})
(\cos(\frac{\theta}{2})I+\sin(\frac{\theta}{2})\hat{n}\cdot\vec{\sigma})=
\cos^2(\frac{\theta}{2})I+\sin^2(\frac{\theta}{2})(\hat{n}\cdot\vec{\sigma})^2=
(\cos^2(\frac{\theta}{2})+\sin^2(\frac{\theta}{2}))I=I$
\subsection*{Exercise 4.6}
First, let's show that $R_Z(x)$ rotates around the Z-axis by an angle $x$.
Consider the general state $\ket{\psi}=\begin{pmatrix}
    \cos\frac{\theta}{2}\\
    e^{i\phi}\sin\frac{\theta}{2}
\end{pmatrix}$. Then,\\
$R_Z(x)\ket{\psi}=(\cos\frac{x}{2}I-i\sin\frac{x}{2}Z)\begin{pmatrix}
    \cos\frac{\theta}{2}\\
    e^{i\phi}\sin\frac{\theta}{2}
\end{pmatrix}=
\cos\frac{x}{2}\begin{pmatrix}
    \cos\frac{\theta}{2}\\
    e^{i\phi}\sin\frac{\theta}{2}
\end{pmatrix}-i\sin\frac{x}{2}\begin{pmatrix}
    \cos\frac{\theta}{2}\\
    -e^{i\phi}\sin\frac{\theta}{2}
\end{pmatrix}=
\begin{pmatrix}
    e^{-ix/2}\cos\frac{\theta}{2}\\
    e^{ix/2}e^{i\phi}\sin\frac{\theta}{2}
\end{pmatrix}=
\begin{pmatrix}
    \cos\frac{\theta}{2}\\
    e^{i(\phi+x)}\sin\frac{\theta}{2}
\end{pmatrix}$\\
Hence, the state has been rotated by $x$ around the Z-axis. Similarly, we get
that $R_X(x)$ and $R_Y(x)$ rotate around the X and Y axis respectively.\\
We also have that,\\
$R_n(x)=\cos\frac{x}{2}I-i\sin\frac{x}{2}(n_xX+n_yY+n_ZZ)=
\cos\frac{x}{2}I-i\sin\frac{x}{2}(\sin\theta_n\cos\phi_nX+\sin\theta_n\sin\theta_nY
+\cos\theta_nZ)=R_Z(\phi_n)R_X(\theta_n)(\cos\frac{x}{2}I-i\sin\frac{x}{2}Z) R_X(\theta_n)^\dagger R_Z(\phi_n)^\dagger=
R_Z(\phi_n)R_X(\theta_n)R_Z(x) R_X(\theta_n)^\dagger R_Z(\phi_n)^\dagger$\\
Therefore, $R_n(x)$ rotates the axis of rotation to the Z axis performs the rotations
by angle $x$ and then returns the axis back to $n$, which is the same as rotating around $n$
by an angle $x$. 
\subsection*{Exercise 4.7}
$\{X,Y\}=1$ therefore, $XYX=-XXY=-Y$.\\
$XR_Y(\theta)X=X(\cos\frac{\theta}{2}I-i\sin\frac{\theta}{2}Y)X=\cos\frac{\theta}{2}I+i\sin\frac{\theta}{2}Y=R_Y(-\theta)$
\subsection*{Exercise 4.8}
Any 2x2 unitary matrix for $a^2+b^2+c^2+d^2=1$ can be written as,\\
1)$U=e^{i\alpha}\begin{bmatrix}
    a+ib&c+id\\
    -c+id& a-ib
\end{bmatrix}$\\
Consider, the given form for $U$,\\
$U=e^{i\alpha}R_n(\theta)=e^{i\alpha}\begin{bmatrix}
    \cos\frac{\theta}{2}-i\sin\frac{\theta}{2}n_z & -\sin\frac{\theta}{2}(n_y+in_x)\\
    \sin\frac{\theta}{2}(n_y-in_x)& \cos\frac{\theta}{2}+i\sin\frac{\theta}{2}n_z
\end{bmatrix}$\\
As, $n_x^2+n_y^2+n_z^2=1$ this has the same form as the general $U$, hence any 
arbitrary 2x2 unitary matrix can be written as $U=e^{i\alpha}R_n(\theta)$.\\
2) $n_z=\frac{1}{\sqrt{2}}$, $n_y=0$, $n_x=\frac{1}{\sqrt{2}}$,
$\alpha=0$ and $\theta=\pi$.\\
3) $n_x,n_y=0$, $n_z=1$, $\alpha =\theta=\frac{\pi}{4}$.
\subsection*{Exercise 4.9}

\end{document}